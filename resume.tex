% LaTeX resume using res.cls
\documentclass[margin]{res}
%\usepackage{helvetica} % uses helvetica postscript font (download helvetica.sty)
%\usepackage{newcent}   % uses new century schoolbook postscript font
\usepackage[colorlinks = true,
            linkcolor = blue,
            urlcolor  = blue,
            citecolor = blue,
            anchorcolor = blue]{hyperref}
\setlength{\textwidth}{5in} % set width of text portion
% \vspace*{-6mm}

\begin{document}
\vspace*{-2.0cm}
% Center the name over the entire width of resume:
 \moveleft.5\hoffset\centerline{\large\bf Karan Agarwal}
 \moveleft.5\hoffset\centerline{\url{http://karanagarwal.me}}
% Draw a horizontal line the whole width of resume:
 \moveleft\hoffset\vbox{\hrule width\resumewidth height 1pt}\smallskip
% address begins here
% Again, the address lines must be centered over entire width of resume:
% \moveleft.5\hoffset\centerline{The LNM Institute of Information Technology}
% \moveleft.5\hoffset\centerline{Sumel, Via-Jamdoli, Jaipur 302031. India}
% \moveleft.5\hoffset\centerline{example@email.com}
% \moveleft.5\hoffset\centerline{+91-123-456-7890}
%

% \noindent\makebox[\textwidth]{example@email.com,example@email.com\hfill +91-123-456-7890}


\begin{resume}

\section{EDUCATION}

\begin{tabular}{|c|c|c|c|}

\hline
\textbf{Degree} & \textbf{Year} & {Institution/Board} & {CPI/\%} \\
\hline
B.Tech & 2019 (Expected) & LNM Institute of Information Technology, Jaipur & 8.35/10 \\
\hline
XII & 2015 & St. Xavier’s Sr. Sec. School, Jaipur (CBSE) & 87.2\% \\
\hline
X & 2013 & St. Xavier’s Sr. Sec. School, Jaipur (CBSE) & 91.2\% \\
\hline
\end{tabular}

\section{PROJECTS}

  {\textbf{Common Profile Builder}} \href{https://github.com/karanagarwal17/web_scrapper_nodejs}{Github} \hfill May `14 - Aug `14\\
  Common profile builder was made.

  {\textbf{Responsive Website}} \href{https://github.com/karanagarwal17/responsive-web-project}{Github} \hfill Dec `16\\
  Responsive Website was developed.

  {\textbf{Comparision of Algorithms}} \href{https://github.com/karanagarwal17/comparision-of-algorithms}{Github} \hfill Dec `16\\
  The efficiency of five algorithms was tested on a randomly generated data set. The number of comparisons required by each algorithm was counted against the size of the input array. Then a graph was plotted using GNU-Plot and Okular. Also, report and presentation were made using Latex.

  {\textbf{Data Validation}} \href{https://github.com/karanagarwal17/Data-Validation}{Github} \hfill Dec `16\\
  Data Validation Project was developed in C.

%  \textbf{More projects on \href{https://github.com/shivamdixit}{Github} account.}

\section{TECHNICAL \\ SKILLS}
  {\textbf{Languages}:} JavaScript, C\\
	{\textbf{Web Development}:} JavaScript, HTML/CSS\\
  {\textbf{Frameworks}:} Node.Js , Express.Js\\
  {\textbf{Databases}:} MongoDB\\
  {\textbf{Libraries Explored}:} Jquery, Bootstrap, Cheerio.Js\\
  {\textbf{Others}:} Git\\
  {\textbf{Platforms}:} GNU/Linux

\section{POSITIONS OF\\ RESPONSIBIL-\\ITY}
    Teaching Assistant (TA) for Computer Programming Course. \hfill July `16 - Dec `16\\

\begin{center}
  \begin{footnotesize}
    Last updated: \today \\
  \end{footnotesize}
\end{center}

\end{resume}
\end{document}
