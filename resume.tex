% LaTeX resume using res.cls
\documentclass[margin]{res}
%\usepackage{helvetica} % uses helvetica postscript font (download helvetica.sty)
%\usepackage{newcent}   % uses new century schoolbook postscript font
\usepackage[colorlinks = true,
            linkcolor = blue,
            urlcolor  = blue,
            citecolor = blue,
            anchorcolor = blue]{hyperref}
\setlength{\textwidth}{5in} % set width of text portion
% \vspace*{-6mm}

\begin{document}
\vspace*{-2.0cm}
% Center the name over the entire width of resume:
 \moveleft.5\hoffset\centerline{\large\bf Karan Agarwal}
 \moveleft.5\hoffset\centerline{\url{http://karanagarwal.me}}
 \moveleft.5\hoffset\centerline{\url{https://github.com/karanagarwal17}}
% Draw a horizontal line the whole width of resume:
 \moveleft\hoffset\vbox{\hrule width\resumewidth height 1pt}\smallskip
% address begins here
% Again, the address lines must be centered over entire width of resume:
% \moveleft.5\hoffset\centerline{The LNM Institute of Information Technology}
% \moveleft.5\hoffset\centerline{Sumel, Via-Jamdoli, Jaipur 302031. India}
% \moveleft.5\hoffset\centerline{example@email.com}
% \moveleft.5\hoffset\centerline{+91-123-456-7890}
%

% \noindent\makebox[\textwidth]{example@email.com,example@email.com\hfill +91-123-456-7890}


\begin{resume}

\section{EDUCATION}

\begin{tabular}{|c|c|c|c|}

\hline
\textbf{Degree} & \textbf{Year} & {Institution/Board} & {CPI/\%} \\
\hline
B.Tech & 2019 (Expected) & LNM Institute of Information Technology, Jaipur & 8.62/10 \\
\hline
XII & 2015 & St. Xavier’s Sr. Sec. School, Jaipur (CBSE) & 87.2\% \\
\hline
X & 2013 & St. Xavier’s Sr. Sec. School, Jaipur (CBSE) & 91.2\% \\
\hline
\end{tabular}

\section{PROJECTS}

  {\textbf{Twitter Live Data Visualization}} \href{https://github.com/karanagarwal17/live-worldmap-visualization}{Github} \hfill May `14 - Aug `14\\
  The Twitter Live Data Visualization project utilized the power of Data Visualizations to create a world-map and uses the Twitter Streaming API to fetch tweets with their location.
  It then plots the tweets on the map according to their co-ordinates and the number of followers the user has.

  {\textbf{Profile Mash}} \href{https://github.com/karanagarwal17/profile-mash}{Github} \hfill May `14 - Aug `14\\
  Developed the Profile Mash where you can enter the links to one or more freelancing platforms and it creates a common profile which can be shown to the client. It solves the problem one faces while shifting to a different platform when all of his work work was on the previous platform.

  {\textbf{Responsive Website}} \href{https://github.com/karanagarwal17/responsive-web-project}{Github} \hfill Dec `16\\
  I made a landing page for an imaginary food delivery start up named Omnifood and learn some aspects of good designing along the way. This is a fully responsive website with which I learned the concepts of flat-UI designing, typography, colour schemes, etc.

  {\textbf{Comparision of Algorithms}} \href{https://github.com/karanagarwal17/comparision-of-algorithms}{Github} \hfill Dec `16\\
  The efficiency of five algorithms was tested on a randomly generated data set. The number of comparisons required by each algorithm was counted against the size of the input array. Then a graph was plotted using GNU-Plot and Okular. Also, report and presentation were made using Latex.

  {\textbf{Data Validation}} \href{https://github.com/karanagarwal17/Data-Validation}{Github} \hfill Dec `16\\
  Data Validation Project was developed in C.

  \textbf{More projects on \href{https://github.com/karanagarwal17}{Github} account.}

  \section{OPEN \\SOURCE \\ PROJECTS}

      \textbf{Mozilla Reps |  Contributor}  | \href{https://github.com/mozilla/remo/pulls/karanagarwal17}{ Github}\hfill 2016 - Present \\
      The Mozilla Reps program aims to empower and support volunteer Mozillians who want to become official representatives of Mozilla in their region/locale. I contributed to the front-end of the platform by fixing bugs.

\section{TECHNICAL \\ SKILLS}
  {\textbf{Languages}:} JavaScript, C\\
	{\textbf{Web Development}:} JavaScript, HTML/CSS\\
  {\textbf{Frameworks}:} Node.Js , Express.Js , Angular.Js\\
  {\textbf{Databases}:} MongoDB, SQL\\
  {\textbf{Libraries Explored}:} D3.Js, jQuery, Bootstrap, Cheerio.Js\\
  {\textbf{Others}:} Git\\
  {\textbf{Platforms}:} Linux

  \section{ACHIEVEME-\\NTS}
  \begin{itemize}  \itemsep -2pt %reduce space between items
      \item Got selected in National Talent Search Examination (NTSE) and rewarded with the prestigious NTSE Scholarship.
      \item Team Member of Team ‘Empyrean’ and Team ‘The Ether Explorer’, both of which made projects for the Space Settlement Design Contest Sponsored by NASA Ames Research Center and bagged Honorable Mention from the same.
      \item Team member of Dougeldyne Astrosystems and Flechtel Constructors which participated in the 10th Annual Asian Regional Space Settlement Design\\ Contest and bagged Runners-up position.
      \item Got selected in Technothlon (twice in a row securing AIR 28 and 35 in the years 2012 and 2013 respectively) conducted by IIT Guwahati and attended Techniche (The Annual Techno-Management Festival of IIT Guwahati) for the same.
  \end{itemize}

  \section{POSITIONS OF\\ RESPONSIBIL-\\ITY}
    Teaching Assistant (TA) for Computer Programming Course. \hfill July `16 - Dec `16\\

\begin{center}
  \begin{footnotesize}
    Last updated: \today \\
  \end{footnotesize}
\end{center}

\end{resume}
\end{document}
